\documentclass[12pt]{report}
\usepackage{graphicx}	
\usepackage{amssymb}
\usepackage{amsmath}
\usepackage{subfiles}
\usepackage{geometry}
\usepackage{braket}
\usepackage{hyperref}
\hypersetup{
    colorlinks=true,
    linkcolor=black,
    filecolor=cyan,      
    urlcolor=cyan,
    pdftitle={Phi Design},
}

\usepackage{listings}
\usepackage{xcolor}

\usepackage{arev}

\usepackage{fontspec}
    \setmainfont{SF Pro Text Regular}
    \setmonofont{Fira Mono}

\usepackage{titlesec}
\titleformat{\chapter}[display]
  {\normalfont\huge}
  {\bfseries\chaptertitlename\ \thechapter}{20pt}{\Huge}

\titleformat{\section}
    {\normalfont\Large}
    {\thesection}{1em}{}

\titleformat{\subsection}
    {\normalfont\itshape\large}
    {\thesubsection}{1em}{}

\definecolor{codegreen}{rgb}{0.3,0.3,0.3}
\definecolor{codegray}{rgb}{0.5,0.5,0.5}
\definecolor{codepurple}{rgb}{0.69,0.69,0.69}
\usepackage{setspace}
\setstretch{2}

\lstdefinestyle{mystyle}{ 
    commentstyle=\color{codegreen},
    keywordstyle=\color{codegreen}\normalfont,
    numberstyle=\tiny\color{codegray},
    stringstyle=\color{codepurple},
    basicstyle=\ttfamily\footnotesize,
    breakatwhitespace=false,         
    breaklines=true,                 
    captionpos=b,                    
    keepspaces=true,                 
    numbers=left,                    
    numbersep=5pt,   
	tabsize=2,
    basicstyle=\ttfamily
}
 
\lstset{style=mystyle}


\usepackage{titling}
\renewcommand\maketitlehooka{\null\mbox{}\vfill}
\renewcommand\maketitlehookd{\vfill\null}

\title{\Huge Romeo and Juliet: Analytical Paragraph}
\author{\LARGE Rishi Kothari}
\date{}

\geometry{margin=1in}

\begin{document}
\pagenumbering{gobble}
\maketitle

\newpage

\pagenumbering{arabic}

In the play 'Romeo and Juliet', by William Shakespeare, the author effectively uses foreshadowing to add substance to the plot, by showing themes of death, violence and conflict to the reader. The plot is introduced in the prologue of the play, and Shakespeare uses the opportunity to forecast two very important deaths in the play. These deaths are foreshadowed by the chorus singing: "The fearful passage of their death-mark'd love" (Prologue 9). The chorus sings of Romeo and Juliet's unfortunate deaths, and their progression of love. The phrase "death mark'd love" shows the reader that their love was doomed to die from the beginning; their love was "marked for death", showing the reader the dark underbelly of the play. Because the theme of death was introduced so early on in the story, the metaphorical stage is set for the reader as a tragedy, allowing Shakespeare to include elements other than romance to the plot of this love story; because the reader gets desensitized to violence in the beginning of the play, Shakespeare has more creative freedom for the plot, allowing it to grow. The author also used foreshadowing to hint at conflict de-escalation, as was shown after Mercutio and Benvolio's fight. At the beginning of the second scene, Capulet states: "But Montague is bound as well as I, In penalty alike; and 'tis not hard, I think, for men as old as we to keep the peace." (I.ii.1-3) This quote shows Capulet having misgivings about the ancient rivalry between the Capulets and Montagues; rather than thinking of the Montagues as their enemies, Capulet is acknowledging them as equals. However, he also makes a reference to a "penalty alike", a clear allusion to the Prince's threat upon Mercutio and Benvolio's lives for fighting a third time in the streets. This reference foreshadows the Prince's threat coming true; the Montagues and Capulets will fight again, and a penalty will be given, but the conflict between families will subside eventually. As a result, this all but requires the author to explore ideas that will create conflict between the families, which thickens the plot. Shakespeare also used foreshadowing to predict violence in the play, shown by the speech of Tybalt at Capulet's party. After Capulet tells Tybalt not to fight Romeo at his party, Tybalt responds under his breath: "I will withdraw, but this intrusion shall; Now seeming sweet, convert to bitterest gall." (I.v.90-91) This foreshadows a fight between Tybalt and Romeo; Tybalt says that the intrusion will lead to his "bitterest gall". This quote shows that Shakespeare is adding events to the plot that create conflict between the Capulets and Montagues, under the guise of violence. Because of this violence, the reader understands that love is not the primary motive for the duelling families; at that moment, the families are still duelling with each other. As a result, Shakespeare can explore this theme further in the play, which adds to the plot. Foreshadowing was a powerful tool in Shakespeare's writing repertoire, and he used this to a large extent in Romeo and Juliet. This idea allowed the author to take themes not unique to love stories, like conflict, violence, and death, and merge them all together to create a plot unlike any other.
\end{document}  